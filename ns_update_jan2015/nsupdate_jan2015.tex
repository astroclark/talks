\documentclass{beamer}
\setbeamertemplate{navigation symbols}{}

\usepackage{beamerthemeshadow}
\setbeamertemplate{caption}[numbered]

\hypersetup{colorlinks}

\def\gw#1{gravitational wave#1 (GW#1)\gdef\gw{GW}}
\def\ns#1{neutron star#1 (NS#1)\gdef\ns{NS}}

\newcommand{\red}[1]{{\color{red}{#1}}}

\begin{document}
\title{NS Group Update}
\subtitle{Burst Call Jan 21$^{\text{st}}$ 2015}  
\author{James A. Clark}
\institute{Georgia Institute Of Technology}
\date{} 

\begin{frame}[plain]
\titlepage
\end{frame}

\begin{frame}\frametitle{Table of contents}\tableofcontents
\end{frame} 

\section{NS Group}

\begin{frame}
    \frametitle{The Group}
    Joining the group:
    \begin{itemize}
        \item Calls: bi-weekly, Friday 11am EST, burst TeamSpeak channel
        \item
            {\small\href{https://wiki.ligo.org/viewauth/GIANT/GIANTteleconAgendas}
            {https://wiki.ligo.org/viewauth/GIANT/GIANTteleconAgendas}}
        %\item Call time negotiable!
        %\item External speakers to resume soon, watch this space\dots
    \end{itemize}
    Scope:
    \begin{itemize}
        \item Explore / discuss astrophysics \& data analysis strategies
            pertaining to transient, unmodelled \gw{} bursts from \ns{s}
    \end{itemize}
    Projects:
    \begin{itemize}
        \item Long bursts: Proto-magnetar deformations \& Magnetar QPOs
        \item Short bursts: Post-BNS \& magnetars
        \item \dots others welcome!
    \end{itemize}
\end{frame}

\section{NS Search Proposal}

\begin{frame}
    \frametitle{NS Search Proposal}
    Search proposal wrapped up and reviewer responses addressed:
    \begin{itemize}
        \item Current draft in DCC: {\small \href{https://dcc.ligo.org/LIGO-T1400606}{https://dcc.ligo.org/LIGO-T1400606}}
        \item Reviewer comments / responses: \url{https://wiki.ligo.org/DAC/NS}
    \end{itemize}
    For O1 plan is to only target \emph{extraordinary} events:
    \begin{itemize}
        \item Hyper-flares from Galactic magnetars (c.f., SGR 1806-20):
            X-pipeline \& STAMP
        \item BNS: long bursts (STAMP / X-pipeline) from long-lived remnant,
            short bursts from short- or long-lived post-merger remnant
    \end{itemize}
    Review comments addressed, only substantive change: short post-merger
    analysis to be PE-style follow-up, not a search.
\end{frame}

\section{Project News}

\subsection{NS Long Bursts}
\begin{frame}
    \frametitle{BNS Long Burst Study}
    Preliminary MDC study to assess sensitivity to long-duration, slightly
    non-stationary signals from stable BNS remnants
    \begin{itemize}
        \item People: Michael Coughlin, Scott Coughlin, James Clark, Ryan
            Quitzow-James, Marie-Anne Bizourd, Nelson Christensen, Patrick
            Meyers, Eric Thrane
        \item Basic idea: BNS merger \emph{may} result in a long-lived, massive
            neutron star; $B$-fields could result in quadrupole deformation
            (e.g., \url{http://arxiv.org/abs/1408.0013})
        \item Signal: anti-chirp starting $\sim$kHz sweeps down in frequency
            over $\mathcal{O}(10^6)$\,s
        \item Optimal search: 10--100\,Mpc / 0.1--1 year$^{-1}$
    \end{itemize}
\end{frame}

\begin{frame}
    \frametitle{BNS Long Burst Study}
    As a preliminary MDC study, considering $\sim$monochromatic signals at 600,
    750 \& 900\,Hz with $\tau\sim250$\,s.\\~\\

    Goals:
    \begin{itemize}
        \item Deploy simulation infrastructure appropriate for long signals:
            using swig-wrapped LAL routines in bespoke python module
        \item Target common set of MDC fr

\end{frame}

\subsection{Post-BNS Bursts}
%\begin{frame}
%    \tableofcontents[currentsection,currentsubsection]
%\end{frame}

\begin{frame}
    \frametitle{LIB Studies}
    \begin{itemize}
        \item
    \end{itemize}
\end{frame}

\begin{frame}
    \frametitle{CWB Studies}
    \begin{itemize}
        \item
    \end{itemize}
\end{frame}


\begin{frame}
    \frametitle{Summary}
\end{frame}




\end{document}
