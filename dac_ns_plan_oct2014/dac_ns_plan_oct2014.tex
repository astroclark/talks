\documentclass{beamer}
\setbeamertemplate{navigation symbols}{}

\usepackage{beamerthemeshadow}
\setbeamertemplate{caption}[numbered]

\hypersetup{colorlinks}

\def\gw#1{gravitational wave#1 (GW#1)\gdef\gw{GW}}
\def\ns#1{neutron star#1 (NS#1)\gdef\ns{NS}}

\newcommand{\red}[1]{{\color{red}{#1}}}

\begin{document}
\title{Neutron Star Search Plan}
\subtitle{DAC Call Oct 24$^{\text{th}}$ 2014}  
\author{James A. Clark For the NS \& Burst Groups}
\institute{Georgia Institute Of Technology}
\date{} 

\begin{frame}[plain]
\titlepage
\end{frame}

%\begin{frame}\frametitle{Table of contents}\tableofcontents
%\end{frame} 


\section{NS Search Proposal Highlights}

\begin{frame}
    \frametitle{NS Search Proposal}
    Search proposal nearing completion:
    \begin{itemize}
        \item Current draft in DCC: {\small \href{https://dcc.ligo.org/LIGO-T1400606}{https://dcc.ligo.org/LIGO-T1400606}}
        \item DAC SVN:
            {\small \href{https://trac.ligo.caltech.edu/dac/browser/WhitePaper/2014-2015}
            {https://trac.ligo.caltech.edu/dac/browser/WhitePaper/2014-2015}}
    \end{itemize}
    For O1 plan is to only target \emph{extraordinary} events:
    \begin{itemize}
        \item Hyper-flares from Galactic magnetars (c.f., SGR 1806-20)
        \item Targeted follow-up for BNS
    \end{itemize}
    Speculative/optimistic sources, but rare occurrences, huge
    science potential, minimal resource requirements \& build off or use
    analyses which are either mature or under development anyway.
\end{frame}

\subsection{Galactic Magnetar Hyperflares}
\begin{frame}
    \frametitle{Galactic Hyperflares: science case}
    \begin{itemize}
        \item LSC/Virgo have a history of high-profile, astrophysically interesting
            searches for \gw{s} associated with magnetar flaring activity

        \item O1 will probably not dramatically improve upon these results for
            normal flaring activity

        \item However, recall SGR 1806-20: $10^{47}$\,erg in hard X-rays / soft $\gamma$-rays in
            $<1$\,s, \emph{within our Galaxy}

        \item An event of this nature would generate significant astrophysical
            interest (i.e., headlines) and may excite various \ns{} oscillation
            modes and/or instabilities; we should be prepared to say something
    \end{itemize}
\end{frame}

\begin{frame}
    \frametitle{Galactic Hyperflares: search method}
    Minimum:
    \begin{itemize}
        \item Immediate examination of burst online triggers
        \item Manual analysis with optimized of \textsc{X-Pipeline}
        \item Minimal development requirements, extends to higher frequencies
            than standard GRB analysis
        \item Only a small chance ($\sim 1\%$) of occurrence
    \end{itemize}
    Nominal:
    \begin{itemize}
        \item STAMP-GRB-like analysis for sensitivity to longer signals (e.g.,
            instabilities, QPOs, \dots)
    \end{itemize}
\end{frame}

\begin{frame}
    \frametitle{Galactic Hyperflares: Publication Plan}
    \begin{itemize}
        \item {\bf Confident or Marginal Detection}: report detection \&
            spectral analysis of waveform 
        \item {\bf No Detection}: only publish if ratio of \gw{} energy U.L. to
            isotropic EM energy is \emph{significantly} lower (better) than all
            past magnetar analyses.  Otherwise, non-detection statement in e.g.,
            end-of-run GRB publication
    \end{itemize}
\end{frame}

\subsection{Post-BNS Bursts}
%\begin{frame}
%    \tableofcontents[currentsection,currentsubsection]
%\end{frame}

\begin{frame}
    \frametitle{Postmerger Bursts: science case}
    \begin{itemize}
        \item Likely outcome of BNS coalescence: formation of (quasi) stable,
            differentially rotating, oscillating \ns{} remnant
        \item 10--100\,ms, 1--4\,kHz \gw{} emission, detectable in aLIGO (ZDHP) to
            few--10's of Mpc 
        \item Detection \& frequency estimation can constrain \ns{} equation of
            state
        \item Not likely to be detectable in O1  BUT astrophysics potential is
            enormous
        \item Also: if the remnant is stable, stronger, longer-lived \gw{}
            emission is possible through e.g., bar-mode instabilities, B-field
            induced quadrupole moment, \dots (10's--100\,Mpc)
        \item Plausible first detection scenario: BNS inspiral; should be
            prepared to answer questions about the post-merger scenario
    \end{itemize}
\end{frame}

\begin{frame}
    \frametitle{Postmerger Bursts: search method}
    \begin{itemize}
        \item Post-merger analysis $\iff$ inspiral detection.
        \item Minimum: correlation of all-sky HF triggers (and reconstructions)
            with inspiral time of coalescence $t_c \pm 100$\,ms
        \item Nominal: high-frequency \textsc{X-Pipeline} search \& targeted
            (narrow time-window around $t_c$, restricted sky-location) follow-up
            of inspiral trigger with burst parameter estimation analysis
        \item Use STAMP-GRB-like analysis to search for longer lived-signals
    \end{itemize}
\end{frame}

\begin{frame}
    \frametitle{Postmerger Bursts: Publication Plans}
    \begin{itemize}
        \item {\bf Confident \& Marginal Detections}: any detection of a
            post-merger BNS signal will be ground-breaking.  A confident
            detection will give higher fidelity reconstruction and spectral
            analysis, but marginal detection alone indicates survival against
            prompt-collapse.  Publish either to accompany CBC deep P.E.
        \item {\bf No detection}: Degeneracy between prompt-collapse \&
            surviving but distant post-merger \ns{} renders upper limits
            somewhat unininteresting; ``no evidence for a post-merger signature
            was observed'' in the inspiral detection paper would suffice.
    \end{itemize}
\end{frame}

\begin{frame}
    \frametitle{NS plans: Resource Requirements}
    \begin{itemize}
        \item Expect $<1$ event to follow-up for magnetars or BNS
        \item Think of as a bonus GRB!
        \item Development to dominate resource requirements (tuning
            time/frequency space; already underway)
    \end{itemize}
\end{frame}


\begin{frame}
    \frametitle{Summary}
    \small{
    \begin{itemize}
        \item NS search plan draft in DCC: {\small \href{https://dcc.ligo.org/LIGO-T1400606}{https://dcc.ligo.org/LIGO-T1400606}}
        \item Targeting NS oscillations \& deformations: $f$-modes, torsional modes,
            instabilities and the unknown
        \item Two trigger types of interest for O1:
            \begin{itemize}
                \item Galactic magnetar hyperflares to trigger: broad-band short burst
                    search ($g$-modes, $f$-modes, unknown) \&
                    long-duration transient search for QPO/torsional modes / instabilities
                \item BNS inspiral \gw{} detection to trigger: high-frequency short-burst
                    search (post-merger $f$-modes) \& long-duration transient search for
                    instabilities or magnetic-field deformations in stable remnants
            \end{itemize}
    \item Rare events, similar to GRB/SNEWS analyses; effectively +1 GRB
    \end{itemize}
    }

\end{frame}




\end{document}
